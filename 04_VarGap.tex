\renewcommand{\writetable}[2]
{ 
    \begin{figure}[H]
    \centering
            \pgfplotstabletypeset
            [
            columns/gap/.style={column name={Gap}, fixed, precision=2},
            columns/N_2/.style={column name={n=2} ,sci,sci zerofill,sci sep align,precision=2,sci superscript},
            columns/N_6/.style={column name={n=6} ,sci,sci zerofill,sci sep align,precision=2,sci superscript},
            columns/N_10/.style={column name={n=10},sci,sci zerofill,sci sep align,precision=2,sci superscript},
            columns/N_14/.style={column name={n=14},sci,sci zerofill,sci sep align,precision=2,sci superscript},
            columns/N_18/.style={column name={n=18},sci,sci zerofill,sci sep align,precision=2,sci superscript},
            columns/N_22/.style={column name={n=22},sci,sci zerofill,sci sep align,precision=2,sci superscript},
            columns/N_26/.style={column name={n=26},sci,sci zerofill,sci sep align,precision=2,sci superscript},
            ]{#1}  
            \caption{#2 x Dimensão do problema.}
    \end{figure}
}

    \subsubsection{Tempo de Processamento para a análise da variação do gap de dualidade para diferentes dimensões para o algoritmo de Pontos Interiores} 
            \writetable{data/data13_tbl_gap_tempo.txt}{Tempo} 

    \subsubsection{Número de iterações para a análise da variação do gap de dualidade para diferentes dimensões para o algoritmo de Pontos Interiores} 
            \writetable{data/data14_tbl_gap_N_iter.txt}{Tempo} 

        \begin{figure}[H]
            \begin{subfigure}{0.5\textwidth}
                \centering
                \begin{tikzpicture}[scale=0.8]
                    \begin{axis}[
                        ymin=-2.7e-3,
                        xlabel={Dimensão},
                        ylabel={Tempo de execução [S]},
                        title={Dimensão x Tempo},
                        legend pos=south east,
                        legend entries={{gap=1e-1},{gap=1e-2},{gap=1e-4},{gap=1e-6},{gap=1e-8},{gap=1e-10}},
                        ymajorgrids=true]
                        \addplot table [x=n,y=0.1]{data/data13b_plt_gap_tempo.txt};
                        \addplot table [x=n,y=0.01]{data/data13b_plt_gap_tempo.txt};
                        \addplot table [x=n,y=0.0001]{data/data13b_plt_gap_tempo.txt};
                        \addplot table [x=n,y=0.000001]{data/data13b_plt_gap_tempo.txt};
                        \addplot table [x=n,y=0.00000001]{data/data13b_plt_gap_tempo.txt};
                        \addplot table [x=n,y=1E-10]{data/data13b_plt_gap_tempo.txt};
                    \end{axis}           
                \end{tikzpicture}
                \caption{ }
                \label{fig:var_gap_tempo_PI}
            \end{subfigure}
            \hfill
            \begin{subfigure}{0.5\textwidth}
                \centering
                \begin{tikzpicture}[scale=0.8]
                    \begin{axis}[
                        ymin=0,
                        xlabel={Dimensão},
                        ylabel={Número de Iterações},
                        title={Dimensão x Iteração},
                        legend pos=south east,
                        legend entries={{gap=1e-1},{gap=1e-2},{gap=1e-4},{gap=1e-6},{gap=1e-8},{gap=1e-10}},
                        ymajorgrids=true]
                        \addplot table [x=n,y=0.1]{data/data14b_plt_gap_N_iter.txt};
                        \addplot table [x=n,y=0.01]{data/data14b_plt_gap_N_iter.txt};
                        \addplot table [x=n,y=0.0001]{data/data14b_plt_gap_N_iter.txt};
                        \addplot table [x=n,y=0.000001]{data/data14b_plt_gap_N_iter.txt};
                        \addplot table [x=n,y=0.00000001]{data/data14b_plt_gap_N_iter.txt};
                        \addplot table [x=n,y=1E-10]{data/data14b_plt_gap_N_iter.txt};
                    \end{axis}           
                \end{tikzpicture}
                \caption{ }
                \label{fig:var_gap_Iter_PI}
            \end{subfigure}        
            \caption{Análise da sensibilidade do gap (a) Tempo de processamento (b) Número de Iterações.}
            \label{fig:var_gap_PI}
        \end{figure}

\subsubsection{Erro percentual da função objetivo para o algoritmo de Pontos Interiores}
    \writetable{data/data20_tbl_gap_err_per_fobj.txt}{Erro percentual da função objetivo}
    
\subsubsection{Erro percentual da solução para o algoritmo de Pontos Interiores}
    \writetable{data/data21_tbl_gap_err_per_sol.txt}{Erro percentual da solução}