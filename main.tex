\documentclass[11pt,a4paper]{article}

\NewDocumentCommand{\nomeAluno}{}{Charles Quirino Pimenta}

\usepackage[brazil]{babel}
\usepackage[T1]{fontenc}
\usepackage[utf8]{inputenc}
\usepackage{lmodern}
\usepackage{ae}
\usepackage{xargs}
\usepackage{indentfirst}
\usepackage[margin=2cm]{geometry}
\usepackage{amsthm,amssymb,amsfonts,amsmath}
\usepackage{psfrag}
\usepackage{booktabs}
\usepackage{multirow}
\usepackage{enumitem}
\usepackage{colortbl,booktabs}
\usepackage{enumerate}
\usepackage{enumitem}
\usepackage{csquotes}
\usepackage{multicol}
% Para plotar tabelas
\usepackage{pgfplots}
\usepackage{pgfplotstable}
\usepackage{booktabs}
\usepackage{array}
\usepackage{colortbl}
\usepackage{booktabs}
\usepackage{array}
\usepackage{placeins}
\usepackage{
tikz,
relsize,
amsmath,
booktabs,
tikz
}
\usepackage{caption}
\usepackage{subcaption}
\usepackage{graphicx}
\usepackage{float}

\pgfplotstableset{% global config, for example in the preamble
  every head row/.style={before row=\toprule,after row=\midrule},
  every last row/.style={after row=\bottomrule},
}
\pgfplotsset{compat=1.18}

\begin{document}
\thispagestyle{empty}

\noindent
\begin{minipage}{0.8\linewidth}
  {\Large\bf Universidade Federal de Minas Gerais}\\
  {\small ESCOLA DE ENGENHARIA}\\
  {\sc Departamento de Engenharia Elétrica}
\end{minipage} 
\hfill 
\begin{minipage}{3cm}
  \includegraphics[width=3cm]{ufmg_ext.pdf}
\end{minipage}

\vspace{1mm}

\noindent
\hrule

\vspace{2.0cm}

\vfill

\begin{center}
  \Large \textsc{\textbf{Trabalho I}}\\ 
  \Large \textsc{\textbf{Otimização em Engenharia}}
\end{center}

\vfill

\begin{center}
  \Large\textsc{\textbf{Método de Pontos Interiores Conjugado com o SIMPLEX para a resolução Do Problema Klee-Minty}}
\end{center}

\vfill

\begin{flushright}
\begin{minipage}{12.0cm}
{\bf Aluno:} \nomeAluno \\
\end{minipage}
\end{flushright}

\vfill

\begin{center}
  \today
\end{center}

\vfill

\newpage 
\section{Experimentos realizados}

    Os algoritmos utilizados para obtenção dos resultados que solucionam os problemas de programação linear complexos são: SIMPLEX, Pontos Interiores e Híbrido do SIMPLEX e PI com Chaveamento via Método de Murty. 
    
    Ressalta-se todos os resultados de tempo de execução foram obtidos por meio dos comandos tic e toc do Matlab\textregistered. Para cada resultado obtido, realizaram-se dez execuções, através das quais foram calculadas as médias do tempo disposto. Isto foi efetuado a fim de minimizar a variabilidade do tempo de execução, devido a interferência dada por outros programas rodando em plano de fundo do computador.
    
    \subsection{Variação da Dimensão no espaço}
    
        Na primeira análise estudou-se os resultados dos algoritmos de solução de problemas de programação linear em relação à variação da dimensão do problema. Nesse caso, os valores dos parâmetros de entrada do método de Pontos Interiores, o gap de dualidade e o passo, foram fixados em $10^{-2}$ e $0.5$ respectivamente.  
        
        Inicialmente avaliou-se o custo computacional de cada um dos algoritmos, considerando problemas de dimensão variando na proporção de quatro em quatro, começando da dimensão n=2 até n=26. Nas tabelas a seguir apresentou-se o tempo de processamento e o número de iterações versus a dimensão do problema para cada algoritmo analisado.

        %% Parte 1
\newcommand{\writetable}[2]
{ 
    \begin{figure}[!ht] 
    \centering
            \pgfplotstabletypeset
            [
                columns/freq/.style={column name=Dim, fixed, precision=1},
                columns/conc/.style={column name=Simplex,sci,sci zerofill,sci sep align,precision=1,sci superscript},
                columns/lino/.style={column name=PI,sci,sci zerofill,sci sep align,precision=1,sci superscript},
                columns/lino/.style={column name=Hibrido,sci,sci zerofill,sci sep align,precision=1,sci superscript},
            ]{#1}  
            \caption{#2 x Dimensão do problema.}
    \end{figure}
}

\newcommand{\plotline}[3]
{
    \begin{tikzpicture}[scale=0.6]
        \begin{axis}[
            xlabel={Dimensão},
            ylabel={Tempo de execução [S]},
            ymode=log,
            title={#3},
            legend pos=south east,
            legend entries={#2},ymajorgrids=true]
            \addplot table [x=Dim,y=#2]{#1};
        \end{axis}           
    \end{tikzpicture}
}

\section{Variação da Dimensão no espaço}
    \subsection{Tempo de execução X Dimensão do Problema}    
        \FloatBarrier
        \def \namefilepath {data/data01_tbl_dim_tempo.txt}
        \writetable{\namefilepath}{Tempo de execução}
        \begin{figure}[h!]
            \begin{subfigure}{0.3\textwidth}
                \centering
                \plotline{\namefilepath}{Simplex}{Tempo x Dimensão}
                \caption{Simplex}
                \label{fig:subim1}
            \end{subfigure}
            \hfill
            \begin{subfigure}{0.3\textwidth}
                \centering
                \plotline{\namefilepath}{PI}{Tempo x Dimensão}
                \caption{PI}
                \label{fig:subim2}
            \end{subfigure}
            \hfill
            \begin{subfigure}{0.3\textwidth}
                \centering
                \plotline{\namefilepath}{Hibrido}{Tempo x Dimensão}
                \caption{Hibrido}
                \label{fig:subim3}
            \end{subfigure}
            \caption{Custo computacional em função da dimensão do problema sob o ponto de vista do tempo de processamento (a) SIMPLEX, (b) Pontos Interiores, (c) Híbrido.}
            \label{fig:image2}
        \end{figure}
        \FloatBarrier
    \subsection{Número de iterações X Dimensão do Problema}    
        \FloatBarrier
        \def \namefilepath {data/data02_tbl_dim_N_Iter.txt}
        \writetable{\namefilepath }{Número de iterações}
        \begin{figure}[h!]
            \begin{subfigure}{0.3\textwidth}
                \centering
                \plotline{\namefilepath }{Simplex}{Iterações x Dimensão}
                \caption{Simplex}
                \label{fig:subim1}
            \end{subfigure}
            \hfill
            \begin{subfigure}{0.3\textwidth}
                \centering
                \plotline{\namefilepath }{PI}{Iterações x Dimensão}
                \caption{PI}
                \label{fig:subim2}
            \end{subfigure}
            \hfill
            \begin{subfigure}{0.3\textwidth}
                \centering
                \plotline{\namefilepath }{Hibrido}{Iterações x Dimensão}
                \caption{Hibrido}
                \label{fig:subim3}
            \end{subfigure}
            \caption{Custo computacional em função da dimensão do problema sob o ponto de vista do número de iterações (a) SIMPLEX, (b) Pontos Interiores, (c) Híbrido.}
            \label{fig:image2}
        \end{figure}        
        \FloatBarrier
    \subsection{Erro percentual da função objetivo X Dimensão do Problema}
        \writetable{data/data03_tbl_err_perc_fobj.txt}{Erro percentual da função objetivo}
    \subsection{Erro percentual da solução X Dimensão do Problema}
        \writetable{data/data04_tbl_err_perc_sol.txt}{Erro percentual da solução}
    \subsection{Conclusão}


    
    \newpage
    
        \subsubsection{Conclusão}

            Observando-se os gráficos da figura \ref{fig:vardim_tempo}, que apresenta o custo computacional em função da dimensão do problema, sob o ponto de vista do tempo de processamento em (s). Pode-se concluir que o tempo de execução do algoritmo Simplex possui relação exponencial ao número de dimensões do problema de Klee-Minty.
            
            Para os algoritmos de Pontos Interiores e Hibrido o mesmo não se confirma. Para estes algoritmos o tempo de execução observado não é sensível ao número de dimensões, assemelhando-se a um valor constante. Mesmo para dimensões cujo tempo de execução se torna proibitivo utilizando o algoritmo Simplex, o algoritmo de Pontos Interiores e o algoritmo Hibrido performam o cálculo no mesmo tempo que levariam para problemas de dimensões de ordem inferior.

            Uma hipótese que talvez possa explicar a variação não linear do tempo de execução do algoritmo em relação ao número de iterações é que maiores dimensões do problema implicam na manipulação de matrizes maiores. 
            
            O mesmo comportamento pode ser observado na figura \ref{fig:vardim_iter} que apresenta o custo computacional em função da dimensão do problema, sob o ponto de vista do número de iterações do algoritmo empregado. Pode-se concluir que o número de iterações do algoritmo Simplex possui relação exponencial ao número de dimensões do problema de Klee-Minty, assim como previsto pelo valor teórico de $N Iter = 2 ^{dim}$. De fato, o tempo de execução é diretamente proporcional ao número de iterações.

            Desta maneira, pode-se observar pela análise da figura \ref{fig:vardim_iter_PI} e \ref{fig:vardim_iter_S} que o número de iterações dos algoritmos de Pontos Interiores e Híbrido converge para um valor constante que não depende do número de dimensões. O algoritmo de Pontos interiores converge para um valor médio de 10 iterações para elevadas dimensões enquanto o algoritmo Híbrido converge para 11 iterações. Isso se deve ao fato do algoritmo Hibrido executar uma única iteração de Simplex após a execução do algoritmo de Pontos Interiores.

            Sob o ponto de vista dos erros percentuais da função objetivo e da solução, é possível concluir pela observação da tabela das figuras 5 e 6, que o algoritmo Simplex sempre zera ambos os erros para qualquer número de dimensões que o problema de Klee-Minty possua. Entretanto, pode-se observar que o algoritmo de Pontos interiores atinge valores de erros tão pequenos quanto se queira. Estes são configuráveis pelo valor escolhido para o gap de dualidade. 

            Para o algoritmo Hibrido é possível notar que este sempre zera ambos os erros com a adição de apenas 1 iteração a mais quando comparado com o algoritmo de pontos interiores.

            Sob todas estas observações é possível afirmar que o algoritmo Hibrido é preferível ao Simplex para a resolução do problema de Klee-Minty para dimensões acima de 2 sob os aspectos considerados de custo computacional, erro da função objetivo e erro da solução.
            
        \newpage
    
    \subsection{Variação do Passo}
    
        Na avaliação do passo, considerou-se o algoritmo de Pontos Interiores e o algoritmo híbrido dos métodos SIMPLEX e PI chaveado por meio do método Murty. Nessa análise, o gap de dualidade foi fixado em $10^{-6}$, enquanto o passo foi variado em valores entre 0 e 1. O gap de dualidade foi fixado em $10^{-6}$ para que o seu valor não influenciasse de forma significativa no estudo da variação do passo. A análise do efeito dessa variação  considerou os diferentes valores de dimensão do problema (n), a fim de avaliar o comportamento dos algoritmos em função da dimensão do problema de Klee-Minty sob o ponto de vista do passo. 
        
        Os resultados obtidos para essa análise, considerando apenas o algoritmo de Pontos Interiores, são apresentados nas tabelas a seguir.

        \renewcommand{\writetable}[2]
{ 
    \begin{figure}[!ht]
    \centering
            \pgfplotstabletypeset
            [
            columns/Passo/.style={fixed, precision=2},
            columns/N_2/.style={column name={n=2} ,sci,sci zerofill,sci sep align,precision=3,sci superscript},
            columns/N_6/.style={column name={n=6} ,sci,sci zerofill,sci sep align,precision=3,sci superscript},
            columns/N_10/.style={column name={n=10},sci,sci zerofill,sci sep align,precision=3,sci superscript},
            columns/N_14/.style={column name={n=14},sci,sci zerofill,sci sep align,precision=3,sci superscript},
            columns/N_18/.style={column name={n=18},sci,sci zerofill,sci sep align,precision=3,sci superscript},
            columns/N_22/.style={column name={n=22},sci,sci zerofill,sci sep align,precision=3,sci superscript},
            columns/N_26/.style={column name={n=26},sci,sci zerofill,sci sep align,precision=3,sci superscript},
            ]{#1}  
            \caption{#2 x Dimensão do problema.}
    \end{figure}
}
 

\section{Variação do Passo}

    \subsection{Tempo de Processamento para a análise da variação do passo para diferentes dimensões para o algoritmo de Pontos Interiores}    
        \writetable{data/data05_tbl_passo_tempo.txt}{Tempo} 
        
    \subsection{Número de iterações para a análise da variação do passo para diferentes dimensões para o algoritmo de Pontos Interiores}    

    \writetable{data/data06_tbl_passo_N_iter.txt}{Número de iterações}

    \FloatBarrier 
        \begin{figure}[h!]
            \begin{subfigure}{0.5\textwidth}
                \centering
                \begin{tikzpicture}[scale=0.8]
                    \begin{axis}[
                        ymin=-2.7e-3,
                        xlabel={Dimensão},
                        ylabel={Tempo de execução [S]},
                        title={Dimensão x Tempo},
                        legend pos=south east,
                        legend entries={{passo=0.1},{passo=0.3},{passo=0.5},{passo=0.7},{passo=0.8},{passo=0.9},{passo=0.95},{passo=0.99}},
                        ymajorgrids=true]
                        \addplot table [x=n,y=0.1]{data/data05b_plt_passo_tempo.txt};
                        \addplot table [x=n,y=0.3]{data/data05b_plt_passo_tempo.txt};
                        \addplot table [x=n,y=0.5]{data/data05b_plt_passo_tempo.txt};
                        \addplot table [x=n,y=0.7]{data/data05b_plt_passo_tempo.txt};
                        \addplot table [x=n,y=0.8]{data/data05b_plt_passo_tempo.txt};
                        \addplot table [x=n,y=0.9]{data/data05b_plt_passo_tempo.txt};
                        \addplot table [x=n,y=0.95]{data/data05b_plt_passo_tempo.txt};
                        \addplot table [x=n,y=0.99]{data/data05b_plt_passo_tempo.txt};
                    \end{axis}           
                \end{tikzpicture}
                \caption{ }
                \label{fig:subim1}
            \end{subfigure}
            \hfill
            \begin{subfigure}{0.5\textwidth}
                \centering
                \begin{tikzpicture}[scale=0.8]
                    \begin{axis}[
                        ymin=0,
                        xlabel={Dimensão},
                        ylabel={Número de Iterações},
                        title={Dimensão x Iteração},
                        legend pos=south east,
                        legend entries={{passo=0.1},{passo=0.3},{passo=0.5},{passo=0.7},{passo=0.8},{passo=0.9},{passo=0.95},{passo=0.99}},
                        ymajorgrids=true]
                        \addplot table [x=n,y=0.1]{data/data06b_plt_passo_N_iter.txt};
                        \addplot table [x=n,y=0.3]{data/data06b_plt_passo_N_iter.txt};
                        \addplot table [x=n,y=0.5]{data/data06b_plt_passo_N_iter.txt};
                        \addplot table [x=n,y=0.7]{data/data06b_plt_passo_N_iter.txt};
                        \addplot table [x=n,y=0.8]{data/data06b_plt_passo_N_iter.txt};
                        \addplot table [x=n,y=0.9]{data/data06b_plt_passo_N_iter.txt};
                        \addplot table [x=n,y=0.95]{data/data06b_plt_passo_N_iter.txt};
                        \addplot table [x=n,y=0.99]{data/data06b_plt_passo_N_iter.txt};
                    \end{axis}           
                \end{tikzpicture}
                \caption{ }
                \label{fig:subim2}
            \end{subfigure}        
            \caption{Análise da sensibilidade do passo (a) Tempo de processamento (b) Número de Iterações.}
            \label{fig:image2}
        \end{figure}
    \FloatBarrier
 
    \subsection{Erro percentual da função objetivo para o algoritmo de Pontos Interiores}
        \writetable{data/data07_tbl_passo_err_perc_fobj.txt}{Erro percentual da função objetivo}
    
    \subsection{Erro percentual da solução para o algoritmo de Pontos Interiores}
         \writetable{data/data08_tbl_passo_err_perc_sol.txt}{Erro percentual da solução}

    \subsection{Tempo de Processamento para a análise da variação do passo para diferentes dimensões para o algoritmo Híbrido} 
     \writetable{data/data09_tbl_passo_tempoH.txt}{Tempo} 

    \subsection{Número de iterações para a análise da variação do passo para diferentes dimensões para o algoritmo Híbrido} 
    \writetable{data/data10_tbl_passo_N_iterH.txt}{Número de iterações} 


    \FloatBarrier 
        \begin{figure}[h!]
            \begin{subfigure}{0.5\textwidth}
                \centering
                \begin{tikzpicture}[scale=0.8]
                    \begin{axis}[
                        ymin=-2.7e-3,
                        xlabel={Dimensão},
                        ylabel={Tempo de execução [S]},
                        title={Dimensão x Tempo},
                        legend pos=south east,
                        legend entries={{passo=0.1},{passo=0.3},{passo=0.5},{passo=0.7},{passo=0.8},{passo=0.9},{passo=0.95},{passo=0.99}},
                        ymajorgrids=true]
                        \addplot table [x=n,y=0.1]{data/data09b_plt_passo_tempoH.txt};
                        \addplot table [x=n,y=0.3]{data/data09b_plt_passo_tempoH.txt};
                        \addplot table [x=n,y=0.5]{data/data09b_plt_passo_tempoH.txt};
                        \addplot table [x=n,y=0.7]{data/data09b_plt_passo_tempoH.txt};
                        \addplot table [x=n,y=0.8]{data/data09b_plt_passo_tempoH.txt};
                        \addplot table [x=n,y=0.9]{data/data09b_plt_passo_tempoH.txt};
                        \addplot table [x=n,y=0.95]{data/data09b_plt_passo_tempoH.txt};
                        \addplot table [x=n,y=0.99]{data/data09b_plt_passo_tempoH.txt};
                    \end{axis}           
                \end{tikzpicture}
                \caption{ }
                \label{fig:subim1}
            \end{subfigure}
            \hfill
            \begin{subfigure}{0.5\textwidth}
                \centering
                \begin{tikzpicture}[scale=0.8]
                    \begin{axis}[
                        ymin=0,
                        xlabel={Dimensão},
                        ylabel={Número de Iterações},
                        title={Dimensão x Iteração},
                        legend pos=south east,
                        legend entries={{passo=0.1},{passo=0.3},{passo=0.5},{passo=0.7},{passo=0.8},{passo=0.9},{passo=0.95},{passo=0.99}},
                        ymajorgrids=true]
                        \addplot table [x=n,y=0.1]{data/data10b_plt_passo_N_iterH.txt};
                        \addplot table [x=n,y=0.3]{data/data10b_plt_passo_N_iterH.txt};
                        \addplot table [x=n,y=0.5]{data/data10b_plt_passo_N_iterH.txt};
                        \addplot table [x=n,y=0.7]{data/data10b_plt_passo_N_iterH.txt};
                        \addplot table [x=n,y=0.8]{data/data10b_plt_passo_N_iterH.txt};
                        \addplot table [x=n,y=0.9]{data/data10b_plt_passo_N_iterH.txt};
                        \addplot table [x=n,y=0.95]{data/data10b_plt_passo_N_iterH.txt};
                        \addplot table [x=n,y=0.99]{data/data10b_plt_passo_N_iterH.txt};
                    \end{axis}           
                \end{tikzpicture}
                \caption{ }
                \label{fig:subim2}
            \end{subfigure}        
            \caption{Análise da sensibilidade do passo (a) Tempo de processamento (b) Número de Iterações.}
            \label{fig:image2}
        \end{figure}
    \FloatBarrier
 
    \subsection{Erro percentual da função objetivo para o algoritmo Híbrido} 
    \writetable{data/data11_tbl_passo_err_perc_fobjH.txt}{Tempo} 
    
    \subsection{Erro percentual da solução para o algoritmo Híbrido}
    \writetable{data/data12_tbl_passo_err_perc_solH.txt}{Tempo} 
    
    \subsection{Conclusão}
    
    \newpage 

        Os resultados obtidos para essa análise, considerando apenas o algoritmo Hibrido, são apresentados nas tabelas a seguir.
        
        \subsection{Tempo de Processamento para a análise da variação do passo para diferentes dimensões para o algoritmo Híbrido} 
     \writetable{data/data09_tbl_passo_tempoH.txt}{Tempo} 

\subsection{Número de iterações para a análise da variação do passo para diferentes dimensões para o algoritmo Híbrido} 
    \writetable{data/data10_tbl_passo_N_iterH.txt}{Número de iterações} 

        \begin{figure}[H]
            \begin{subfigure}{0.5\textwidth}
                \centering
                \begin{tikzpicture}[scale=0.8]
                    \begin{axis}[
                        ymin=-2.7e-3,
                        xlabel={Dimensão},
                        ylabel={Tempo de execução [S]},
                        title={Dimensão x Tempo},
                        legend pos=south east,
                        legend entries={{passo=0.1},{passo=0.3},{passo=0.5},{passo=0.7},{passo=0.8},{passo=0.9},{passo=0.95},{passo=0.99}},
                        ymajorgrids=true]
                        \addplot table [x=n,y=0.1]{data/data09b_plt_passo_tempoH.txt};
                        \addplot table [x=n,y=0.3]{data/data09b_plt_passo_tempoH.txt};
                        \addplot table [x=n,y=0.5]{data/data09b_plt_passo_tempoH.txt};
                        \addplot table [x=n,y=0.7]{data/data09b_plt_passo_tempoH.txt};
                        \addplot table [x=n,y=0.8]{data/data09b_plt_passo_tempoH.txt};
                        \addplot table [x=n,y=0.9]{data/data09b_plt_passo_tempoH.txt};
                        \addplot table [x=n,y=0.95]{data/data09b_plt_passo_tempoH.txt};
                        \addplot table [x=n,y=0.99]{data/data09b_plt_passo_tempoH.txt};
                    \end{axis}           
                \end{tikzpicture}
                \caption{ }
                %\label{fig:subim1}
            \end{subfigure}
            \hfill
            \begin{subfigure}{0.5\textwidth}
                \centering
                \begin{tikzpicture}[scale=0.8]
                    \begin{axis}[
                        ymin=0,
                        xlabel={Dimensão},
                        ylabel={Número de Iterações},
                        title={Dimensão x Iteração},
                        legend pos=south east,
                        legend entries={{passo=0.1},{passo=0.3},{passo=0.5},{passo=0.7},{passo=0.8},{passo=0.9},{passo=0.95},{passo=0.99}},
                        ymajorgrids=true]
                        \addplot table [x=n,y=0.1]{data/data10b_plt_passo_N_iterH.txt};
                        \addplot table [x=n,y=0.3]{data/data10b_plt_passo_N_iterH.txt};
                        \addplot table [x=n,y=0.5]{data/data10b_plt_passo_N_iterH.txt};
                        \addplot table [x=n,y=0.7]{data/data10b_plt_passo_N_iterH.txt};
                        \addplot table [x=n,y=0.8]{data/data10b_plt_passo_N_iterH.txt};
                        \addplot table [x=n,y=0.9]{data/data10b_plt_passo_N_iterH.txt};
                        \addplot table [x=n,y=0.95]{data/data10b_plt_passo_N_iterH.txt};
                        \addplot table [x=n,y=0.99]{data/data10b_plt_passo_N_iterH.txt};
                    \end{axis}           
                \end{tikzpicture}
                \caption{ }
               % \label{fig:subim2}
            \end{subfigure}        
            \caption{Análise da sensibilidade do passo (a) Tempo de processamento (b) Número de Iterações.}
           % \label{fig:image2}
        \end{figure}
 
\subsection{Erro percentual da função objetivo para o algoritmo Híbrido} 
    \writetable{data/data11_tbl_passo_err_perc_fobjH.txt}{Tempo} 
    
\subsection{Erro percentual da solução para o algoritmo Híbrido}
    \writetable{data/data12_tbl_passo_err_perc_solH.txt}{Tempo} 

        \subsubsection{Conclusão}
        
            Observando-se os gráficos da figura \ref{fig:varpasso_tempo_PI}, que apresenta o custo computacional em função do tamanho do passo do algoritmo de pontos Interiores, para diferentes dimensões do problema e sob o ponto de vista do tempo de processamento em (s). Pode-se concluir que o tempo de execução do algoritmo de Pontos Interiores é pouco sensível a variação do passo para diferentes valores do número de dimensões do problema de Klee-Minty.
            
            Observando-se os gráficos da figura \ref{fig:varpasso_Iter_PI}, que apresenta o custo computacional em função do tamanho do passo do algoritmo de pontos Interiores, para diferentes dimensões do problema e sob o ponto de vista do número de iterações do algoritmo empregado. Pode-se concluir que o número de iterações do algoritmo de Pontos Interiores é pouco sensível a variação do passo para diferentes valores do número de dimensões do problema de Klee-Minty. Na verdade é constante com relação ao tamanho do passo.

            Uma hipótese que talvez possa explicar a variação não linear do tempo de execução do algoritmo em relação ao número de iterações é que maiores dimensões do problema implicam na manipulação de matrizes maiores.

            Sob o ponto de vista dos erros percentuais da função objetivo e da solução, é possível concluir pela observação da tabela das figuras 10 e 11, que o algoritmo de Pontos Interiores não é capaz de zerar ambos os erros e que esta incapacidade não depende do tamanho do passo para problemas de dimensão até $n=26$.  

            Para o algoritmo Hibrido é possível notar que este possui quase as mesmas características do algoritmo de Pontos interiores no que diz respeito ao tempo de execução e número de iterações. O valor do passo escolhido não interfere nestas variáveis sensivelmente. Em relação aos erros da função objetivo e da solução, o passo não interfere na capacidade do algoritmo hibrido de zerar ambos os erros com a adição de apenas 1 iteração a mais quando comparado com o algoritmo de pontos interiores.

            Sob todas estas observações é possível afirmar que os algoritmos de Pontos interiores e Hibrido são pouco sensíveis a variação do passo para a resolução do problema de Klee-Minty. Ambos mantém suas características de desempenho atreladas ao valor do número de dimensões.
        
        %\newpage
    
    \subsection{Variação do Gap de Dualidade}

        Na análise dessa seção, avaliou-se o esforço computacional quanto a variação do gap de dualidade para os algoritmos PI e Híbrido. Nessa análise o passo é fixado em $0.95$ e são considerados diferentes valores de dimensão (n). Além disso, o gap de dualidade varia entre os valores $10^{-1}$ e $10^{-10}$.
        
        Os resultados para a análise da sensibilidade do gap de dualidade referentes à execução do algoritmo de Pontos Interiores, são apresentados nas tabelas a seguir.

        \renewcommand{\writetable}[2]
{ 
    \begin{figure}[H]
    \centering
            \pgfplotstabletypeset
            [
            columns/gap/.style={column name={Gap}, fixed, precision=2},
            columns/N_2/.style={column name={n=2} ,sci,sci zerofill,sci sep align,precision=2,sci superscript},
            columns/N_6/.style={column name={n=6} ,sci,sci zerofill,sci sep align,precision=2,sci superscript},
            columns/N_10/.style={column name={n=10},sci,sci zerofill,sci sep align,precision=2,sci superscript},
            columns/N_14/.style={column name={n=14},sci,sci zerofill,sci sep align,precision=2,sci superscript},
            columns/N_18/.style={column name={n=18},sci,sci zerofill,sci sep align,precision=2,sci superscript},
            columns/N_22/.style={column name={n=22},sci,sci zerofill,sci sep align,precision=2,sci superscript},
            columns/N_26/.style={column name={n=26},sci,sci zerofill,sci sep align,precision=2,sci superscript},
            ]{#1}  
            \caption{#2 x Dimensão do problema.}
    \end{figure}
}

    \subsubsection{Tempo de Processamento para a análise da variação do gap de dualidade para diferentes dimensões para o algoritmo de Pontos Interiores} 
            \writetable{data/data13_tbl_gap_tempo.txt}{Tempo} 

    \subsubsection{Número de iterações para a análise da variação do gap de dualidade para diferentes dimensões para o algoritmo de Pontos Interiores} 
            \writetable{data/data14_tbl_gap_N_iter.txt}{Tempo} 

        \begin{figure}[H]
            \begin{subfigure}{0.5\textwidth}
                \centering
                \begin{tikzpicture}[scale=0.8]
                    \begin{axis}[
                        ymin=-2.7e-3,
                        xlabel={Dimensão},
                        ylabel={Tempo de execução [S]},
                        title={Dimensão x Tempo},
                        legend pos=south east,
                        legend entries={{gap=1e-1},{gap=1e-2},{gap=1e-4},{gap=1e-6},{gap=1e-8},{gap=1e-10}},
                        ymajorgrids=true]
                        \addplot table [x=n,y=0.1]{data/data13b_plt_gap_tempo.txt};
                        \addplot table [x=n,y=0.01]{data/data13b_plt_gap_tempo.txt};
                        \addplot table [x=n,y=0.0001]{data/data13b_plt_gap_tempo.txt};
                        \addplot table [x=n,y=0.000001]{data/data13b_plt_gap_tempo.txt};
                        \addplot table [x=n,y=0.00000001]{data/data13b_plt_gap_tempo.txt};
                        \addplot table [x=n,y=1E-10]{data/data13b_plt_gap_tempo.txt};
                    \end{axis}           
                \end{tikzpicture}
                \caption{ }
                \label{fig:var_gap_tempo_PI}
            \end{subfigure}
            \hfill
            \begin{subfigure}{0.5\textwidth}
                \centering
                \begin{tikzpicture}[scale=0.8]
                    \begin{axis}[
                        ymin=0,
                        xlabel={Dimensão},
                        ylabel={Número de Iterações},
                        title={Dimensão x Iteração},
                        legend pos=south east,
                        legend entries={{gap=1e-1},{gap=1e-2},{gap=1e-4},{gap=1e-6},{gap=1e-8},{gap=1e-10}},
                        ymajorgrids=true]
                        \addplot table [x=n,y=0.1]{data/data14b_plt_gap_N_iter.txt};
                        \addplot table [x=n,y=0.01]{data/data14b_plt_gap_N_iter.txt};
                        \addplot table [x=n,y=0.0001]{data/data14b_plt_gap_N_iter.txt};
                        \addplot table [x=n,y=0.000001]{data/data14b_plt_gap_N_iter.txt};
                        \addplot table [x=n,y=0.00000001]{data/data14b_plt_gap_N_iter.txt};
                        \addplot table [x=n,y=1E-10]{data/data14b_plt_gap_N_iter.txt};
                    \end{axis}           
                \end{tikzpicture}
                \caption{ }
                \label{fig:var_gap_Iter_PI}
            \end{subfigure}        
            \caption{Análise da sensibilidade do gap (a) Tempo de processamento (b) Número de Iterações.}
            \label{fig:var_gap_PI}
        \end{figure}

\subsubsection{Erro percentual da função objetivo para o algoritmo de Pontos Interiores}
    \writetable{data/data20_tbl_gap_err_per_fobj.txt}{Erro percentual da função objetivo}
    
\subsubsection{Erro percentual da solução para o algoritmo de Pontos Interiores}
    \writetable{data/data21_tbl_gap_err_per_sol.txt}{Erro percentual da solução}

        Os resultados para a análise da sensibilidade do gap de dualidade referentes à execução do algoritmo Híbrido, são apresentados nas tabelas a seguir.

         \subsubsection{Tempo de Processamento para a análise da variação do gap de dualidade para diferentes dimensões para o algoritmo Híbrido} 
    \writetable{data/data17_tbl_gap_tempoH.txt}{Tempo} 

\subsubsection{Número de iterações para a análise da variação do gap de dualidade para diferentes dimensões para o algoritmo Híbrido} 
    \writetable{data/data18_tbl_gap_N_iterH.txt}{Tempo} 

        \begin{figure}[H]
            \begin{subfigure}{0.5\textwidth}
                \centering
                \begin{tikzpicture}[scale=0.8]
                    \begin{axis}[
                        ymin=-2.7e-3,
                        xlabel={Dimensão},
                        ylabel={Tempo de execução [S]},
                        title={Dimensão x Tempo},
                        legend pos=south east,
                        legend entries={{gap=1e-1},{gap=1e-2},{gap=1e-4},{gap=1e-6},{gap=1e-8},{gap=1e-10}},
                        ymajorgrids=true]
                        \addplot table [x=n,y=0.1]{data/data17b_plt_gap_tempoH.txt};
                        \addplot table [x=n,y=0.01]{data/data17b_plt_gap_tempoH.txt};
                        \addplot table [x=n,y=0.0001]{data/data17b_plt_gap_tempoH.txt};
                        \addplot table [x=n,y=0.000001]{data/data17b_plt_gap_tempoH.txt};
                        \addplot table [x=n,y=0.00000001]{data/data17b_plt_gap_tempoH.txt};
                        \addplot table [x=n,y=1E-10]{data/data17b_plt_gap_tempoH.txt};
                    \end{axis}           
                \end{tikzpicture}
                \caption{ }
                \label{fig:var_gap_tempo_H}
            \end{subfigure}
            \hfill
            \begin{subfigure}{0.5\textwidth}
                \centering
                \begin{tikzpicture}[scale=0.8]
                    \begin{axis}[
                        ymin=0,
                        xlabel={Dimensão},
                        ylabel={Número de Iterações},
                        title={Dimensão x Iteração},
                        legend pos=south east,
                        legend entries={{gap=1e-1},{gap=1e-2},{gap=1e-4},{gap=1e-6},{gap=1e-8},{gap=1e-10}},
                        ymajorgrids=true]
                        \addplot table [x=n,y=0.1]{data/data18b_plt_gap_N_iterH.txt};
                        \addplot table [x=n,y=0.01]{data/data18b_plt_gap_N_iterH.txt};
                        \addplot table [x=n,y=0.0001]{data/data18b_plt_gap_N_iterH.txt};
                        \addplot table [x=n,y=0.000001]{data/data18b_plt_gap_N_iterH.txt};
                        \addplot table [x=n,y=0.00000001]{data/data18b_plt_gap_N_iterH.txt};
                        \addplot table [x=n,y=1E-10]{data/data18b_plt_gap_N_iterH.txt};
                    \end{axis}           
                \end{tikzpicture}
                \caption{ }
                \label{fig:var_gap_iter_H}
            \end{subfigure}        
            \caption{Análise da sensibilidade do passo (a) Tempo de processamento (b) Número de Iterações.}
            %\label{fig:image2}
        \end{figure}

\subsection{Erro percentual da função objetivo para o algoritmo Híbrido} 
    \writetable{data/data22_tbl_gap_err_per_fobjH.txt}{Tempo} 
    
\subsection{Erro percentual da solução para o algoritmo Híbrido}
    \writetable{data/data23_tbl_gap_err_per_solH.txt}{Tempo} 
        
        \subsubsection{Conclusão}

            Observando-se os gráficos da figura \ref{fig:var_gap_tempo_PI}, que apresenta o custo computacional em função do gap de dualidade do algoritmo de pontos Interiores, para diferentes dimensões do problema e sob o ponto de vista do tempo de processamento em (s). Pode-se concluir que o tempo de execução do algoritmo de Pontos Interiores é sensível a valores de gap muito pequenos para todos valores do número de dimensões do problema de Klee-Minty. 
            
            Observando-se os gráficos da figura \ref{fig:var_gap_Iter_PI}, que apresenta o custo computacional em função do gap de dualidade do algoritmo de pontos Interiores, para diferentes dimensões do problema e sob o ponto de vista do número de iterações do algoritmo empregado. Pode-se concluir que o número de iterações do algoritmo Híbrido é sensível a valores de gap muito pequenos para todos valores do número de dimensões do problema de Klee-Minty. Para valores de gap maiores ou iguais a $1e-2$ há uma convergência do número de iterações e este volta a ser dependente unicamente do número de dimensões.
 
            Sob o ponto de vista dos erros percentuais da função objetivo e da solução, é possível concluir pela observação da tabela das figuras 20 e 21, que o algoritmo de Pontos Interiores não é capaz de zerar ambos os erros e que esta incapacidade não depende do gap de dualidade. Estes erros estão relacionados apenas ao valor da dimensão.  

            Para o algoritmo Hibrido é possível notar que este possui quase as mesmas características do algoritmo de Pontos interiores no que diz respeito ao tempo de execução e número de iterações. O valor do gap escolhido não interfere nestas variáveis sensivelmente, exceto para valores de gap menores que $1e-2$.  Em relação aos erros da função objetivo e da solução, o passo não interfere na capacidade do algoritmo hibrido de zerar ambos os erros com a adição de apenas 1 iteração a mais quando comparado com o algoritmo de pontos interiores.

            Sob todas estas observações é possível afirmar que, exceto para valores de gap muito pequenos, menores que $1e-2$, os algoritmos de Pontos interiores e Hibrido são pouco sensíveis a variação do passo para a resolução do problema de Klee-Minty. Ambos mantém suas características de desempenho atreladas ao valor do número de dimensões.

        \newpage
        
    \subsection{Análise do chaveamento entre os métodos PI e SIMPLEX}
    
        A grande questão relacionada ao algoritmo Híbrido é quando se deve chavear do algoritmo PI para o SIMPLEX. No intuito de analisar esse chaveamento deve-se realizar um melhor detalhamento quanto aos resultados relacionados ao esforço computacional, considerando as soluções obtidas antes e depois do chaveamento via método de Murty. Dessa forma, os experimentos devem considerar a execução do algoritmo Híbrido para diferentes dimensões, gap de dualidade fixado em a $10^{-6}$ e passo igual a $0.95$. Os resultados desses experimentos serão dispostos na Tabela a seguir:
        
        \begin{figure}[H] 
    \centering
    \pgfplotstabletypeset[
        column type=l,
        every head row/.style={
            before row={
            \toprule
            \multirow{2}{*}{ } & \multicolumn{2}{c}{Iterações} & \multicolumn{2}{c}{Tempo} & \multicolumn{2}{c}{Erro Função obj (\%)} & \multicolumn{2}{c}{Erro Solução (\%)} \\
            },
        after row=\midrule,
        },
        every last row/.style={
            after row=\bottomrule
        },
        columns/Dim/.style ={column name=Dim, fixed, precision=2},
        columns/IteraçõesSimplex/.style ={column name=Simplex, fixed, precision=2},
        columns/IteraçõesPI/.style ={column name=PI, fixed, precision=2},
        columns/TempoSimplex/.style ={column name=Simplex},
        columns/TempoPI/.style ={column name=PI},
        columns/ErroFobjSimplex/.style ={column name=Simplex},
        columns/ErroFobjPI/.style ={column name=PI,sci,sci zerofill,sci sep align,precision=2,sci superscript},
        columns/ErroSolSimplex/.style ={column name=Simplex},
        columns/ErroSolPI/.style ={column name=PI,sci,sci zerofill,sci sep align,precision=2,sci superscript},
        string type,
    ]{data/data19_tbl_chaveamento.txt}
            \caption{Resultados relacionados ao esforço computacional para o algoritmo Híbrido (Detalhamento dos métodos de Pontos Interiores e SIMPLEX).}
    \end{figure}

\begin{figure}[H]  
  \begin{minipage}[b]{0.5\linewidth}
    \begin{tikzpicture}[scale=0.8]
        \begin{axis}[
            xlabel={Dimensão},
            ylabel={Iterações},
            ymode=log,
            %title={Número de iterações},
            legend pos=north west,
            legend entries={{Simplex},{PI}},ymajorgrids=true]
            \addplot table [x=Dim,y=IteraçõesSimplex]{data/data19_tbl_chaveamento.txt};
            \addplot table [x=Dim,y=IteraçõesPI]{data/data19_tbl_chaveamento.txt};
        \end{axis}           
    \end{tikzpicture} 
    \caption{Dimensão x Iteração} 
  \end{minipage} 
  \begin{minipage}[b]{0.5\linewidth}
    \begin{tikzpicture}[scale=0.8]
        \begin{axis}[
            xlabel={Dimensão},
            ylabel={Tempo [S]},
            ymode=log,
            %title={Tempo de execução},
            legend pos=north west,
            legend entries={{Simplex},{PI}},ymajorgrids=true]
            \addplot table [x=Dim,y=TempoSimplex]{data/data19_tbl_chaveamento.txt};
            \addplot table [x=Dim,y=TempoPI]{data/data19_tbl_chaveamento.txt};
        \end{axis}           
    \end{tikzpicture}
    \caption{Dimensão x Tempo} 
  \end{minipage} 
  \begin{minipage}[b]{0.5\linewidth}
    \begin{tikzpicture}[scale=0.8]
        \begin{axis}[
            xlabel={Dimensão},
            ylabel={Erro [\%]},
            ymax=10e-8,
            %title={Erro \% da Função Objetivo},
            legend pos=north west,
            legend entries={{Simplex},{PI}},ymajorgrids=true]
            \addplot table [x=Dim,y=ErroFobjSimplex]{data/data19_tbl_chaveamento.txt};
            \addplot table [x=Dim,y=ErroFobjPI]{data/data19_tbl_chaveamento.txt};
        \end{axis}           
    \end{tikzpicture}[scale=0.8]
    \caption{Dimensão x Erro \% da Função Objetivo} 
  \end{minipage}
  \hfill
  \begin{minipage}[b]{0.5\linewidth}
        \begin{tikzpicture}[scale=0.8]
        \begin{axis}[
            xlabel={Dimensão},
            ylabel={Erro [\%]},
            ymax=10e-8,
            %title={Erro \% da Solução},
            legend pos=north west,
            legend entries={{Simplex},{PI}},ymajorgrids=true]
            \addplot table [x=Dim,y=ErroSolSimplex]{data/data19_tbl_chaveamento.txt};
            \addplot table [x=Dim,y=ErroSolPI]{data/data19_tbl_chaveamento.txt};
        \end{axis}           
    \end{tikzpicture}  
    \caption{Dimensão x Erro \% da Solução} 
  \end{minipage} 
\end{figure}
        
        \subsubsection{Conclusão}

        Sob o estrito ponto de vista de custo computacional, levando em conta apenas o tempo de execução do algoritmo em função do número de dimensões do problema de Klee-Minty. A escolha mais robusta do ponto de chaveamento entre os algoritmos Simplex e Pontos interiores acontece quando o número de dimensões do problema é maior ou igual a $n=6$. Tendo em vista que o tempo de execução e o número de iterações para este algoritmo permanece praticamente constante para qualquer valor do número de dimensões e é sempre menor ou igual aos seus correspondentes para o algoritmo simplex a partir de $n=6$.

        Considerando também que o algoritmo de Pontos interiores não zera absolutamente o erro da função objetivo e da solução, é interessante utilizar o algoritmo hibrido a partir deste ponto para garantir a convergência absoluta da solução e da função objetivo para os seus valores exatos.

        \newpage

    \subsection{Análise de sensibilidade quanto ao parâmetro Valb}

     Na avaliação do parâmetro Valb, considerou-se o algoritmo Simplex, o algoritmo de Pontos Interiores e o algoritmo híbrido dos métodos SIMPLEX e PI chaveado por meio do método Murty. Nessa análise, o gap de dualidade foi fixado em $10^{-6}$, e o passo foi fixado em $0.95$. Enquanto o valor do parâmetro valb foi variado em valores entre 10 e 100 em escala logarítmica.A análise do efeito dessa variação  considerou os diferentes valores de dimensão do problema (n), a fim de avaliar o comportamento dos algoritmos em função da dimensão do problema de Klee-Minty sob o ponto de vista do parâmetro Valb. 
        
        
        
        \subsubsection{Tempo de Processamento para a análise da variação do parâmetro valb para diferentes dimensões}    

\begin{figure}[H] 
    \centering
    \pgfplotstabletypeset
        [
            columns/valb/.style={column name=valb, fixed, precision=2},
            columns/2/.style={column name={n=2},sci,sci zerofill,sci sep align,precision=2,sci superscript},
            columns/4/.style={column name={n=4},sci,sci zerofill,sci sep align,precision=2,sci superscript},
            columns/6/.style={column name={n=6},sci,sci zerofill,sci sep align,precision=2,sci superscript},
            columns/10/.style={column name={n=10},sci,sci zerofill,sci sep align,precision=2,sci superscript},
            columns/14/.style={column name={n=14},sci,sci zerofill,sci sep align,precision=2,sci superscript},
            columns/18/.style={column name={n=18},sci,sci zerofill,sci sep align,precision=2,sci superscript},
            columns/20/.style={column name={n=20},sci,sci zerofill,sci sep align,precision=2,sci superscript},
            columns/22/.style={column name={n=22},sci,sci zerofill,sci sep align,precision=2,sci superscript},
            columns/24/.style={column name={n=24},sci,sci zerofill,sci sep align,precision=2,sci superscript},
        ]{datavalb/01_valb_time_Simplex.txt}  
    \caption{Tempo de execução x Dimensão do problema para o algoritmo Simplex.}
\end{figure}

\begin{figure}[H] 
    \centering
    \pgfplotstabletypeset
        [
            columns/valb/.style={column name=valb, fixed, precision=2},
            columns/2/.style={column name={n=2},sci,sci zerofill,sci sep align,precision=2,sci superscript},
            columns/4/.style={column name={n=4},sci,sci zerofill,sci sep align,precision=2,sci superscript},
            columns/6/.style={column name={n=6},sci,sci zerofill,sci sep align,precision=2,sci superscript},
            columns/10/.style={column name={n=10},sci,sci zerofill,sci sep align,precision=2,sci superscript},
            columns/14/.style={column name={n=14},sci,sci zerofill,sci sep align,precision=2,sci superscript},
            columns/18/.style={column name={n=18},sci,sci zerofill,sci sep align,precision=2,sci superscript},
            columns/20/.style={column name={n=20},sci,sci zerofill,sci sep align,precision=2,sci superscript},
            columns/22/.style={column name={n=22},sci,sci zerofill,sci sep align,precision=2,sci superscript},
            columns/24/.style={column name={n=24},sci,sci zerofill,sci sep align,precision=2,sci superscript},
        ]{datavalb/01_valb_time_PI.txt}  
    \caption{Tempo de execução x Dimensão do problema para o algoritmo de Pontos Interiores.}
\end{figure}

\begin{figure}[H] 
    \centering
    \pgfplotstabletypeset
        [
            columns/valb/.style={column name=valb, fixed, precision=2},
            columns/2/.style={column name={n=2},sci,sci zerofill,sci sep align,precision=2,sci superscript},
            columns/4/.style={column name={n=4},sci,sci zerofill,sci sep align,precision=2,sci superscript},
            columns/6/.style={column name={n=6},sci,sci zerofill,sci sep align,precision=2,sci superscript},
            columns/10/.style={column name={n=10},sci,sci zerofill,sci sep align,precision=2,sci superscript},
            columns/14/.style={column name={n=14},sci,sci zerofill,sci sep align,precision=2,sci superscript},
            columns/18/.style={column name={n=18},sci,sci zerofill,sci sep align,precision=2,sci superscript},
            columns/20/.style={column name={n=20},sci,sci zerofill,sci sep align,precision=2,sci superscript},
            columns/22/.style={column name={n=22},sci,sci zerofill,sci sep align,precision=2,sci superscript},
            columns/24/.style={column name={n=24},sci,sci zerofill,sci sep align,precision=2,sci superscript},
        ]{datavalb/01_valb_time_Hibrido.txt}  
    \caption{Tempo de execução x Dimensão do problema para o algoritmo Híbrido.}
\end{figure}
 
\begin{figure}[H]
    \begin{subfigure}{0.45\textwidth}
        %\centering
        \def\filename{datavalb/01_valb_time_Simplex.txt}  
        \begin{tikzpicture}[scale=0.85]
            \begin{axis}[
                xlabel={valb},
                ylabel={Tempo de execução [S]},
                title={valb x Tempo},
                ymode=log,
                ymax=10^3,
                legend pos=outer north east,
                legend entries={{n=2},{n=4},{n=6},{n=10},{n=14},{n=18},{n=20},{n=22},{n=24}},
                ymajorgrids=true]
                \addplot table [x=valb,y=2]\filename ;
                \addplot table [x=valb,y=4]\filename ;
                \addplot table [x=valb,y=6]\filename ;
                \addplot table [x=valb,y=10]\filename ;
                \addplot table [x=valb,y=14]\filename ;
                \addplot table [x=valb,y=18]\filename ;
                \addplot table [x=valb,y=20]\filename ;
                \addplot table [x=valb,y=22]\filename ;
                \addplot table [x=valb,y=24]\filename ;
            \end{axis} 
        \end{tikzpicture}
        \caption{Simplex}
        \label{fig:vardim_tempo_S}
    \end{subfigure}
    \hfill
    \centering
    \begin{subfigure}{0.45\textwidth}
       % \centering
        \def\filename{datavalb/01_valb_time_PI.txt}  
        \begin{tikzpicture}[scale=0.85] 
            \begin{axis}[
                xlabel={valb},
                ylabel={Tempo de execução [S]},
                title={valb x Tempo},
                ymode=log,
                %ymax=10^3,
                legend pos=outer north east,
                legend entries={{n=2},{n=4},{n=6},{n=10},{n=14},{n=18},{n=20},{n=22},{n=24}},
                ymajorgrids=true]
                \addplot table [x=valb,y=2]\filename ;
                \addplot table [x=valb,y=4]\filename ;
                \addplot table [x=valb,y=6]\filename ;
                \addplot table [x=valb,y=10]\filename ;
                \addplot table [x=valb,y=14]\filename ;
                \addplot table [x=valb,y=18]\filename ;
                \addplot table [x=valb,y=20]\filename ;
                \addplot table [x=valb,y=22]\filename ;
                \addplot table [x=valb,y=24]\filename ;
            \end{axis} 
        \end{tikzpicture}
        \caption{PI}
        \label{fig:vardim_tempo_PI}
    \end{subfigure}
    \vfill
    \begin{subfigure}{0.45\textwidth}
        % \centering
        \def\filename{datavalb/01_valb_time_Hibrido.txt}  
        \begin{tikzpicture}[scale=0.85] 
            \begin{axis}[
                xlabel={valb},
                ylabel={Tempo de execução [S]},
                title={valb x Tempo},
                ymode=log,
                %ymax=10^3,
                legend pos=outer north east,
                legend entries={{n=2},{n=4},{n=6},{n=10},{n=14},{n=18},{n=20},{n=22},{n=24}},
                ymajorgrids=true]
                \addplot table [x=valb,y=2]\filename ;
                \addplot table [x=valb,y=4]\filename ;
                \addplot table [x=valb,y=6]\filename ;
                \addplot table [x=valb,y=10]\filename ;
                \addplot table [x=valb,y=14]\filename ;
                \addplot table [x=valb,y=18]\filename ;
                \addplot table [x=valb,y=20]\filename ;
                \addplot table [x=valb,y=22]\filename ;
                \addplot table [x=valb,y=24]\filename ;
            \end{axis} 
        \end{tikzpicture}
        \caption{Hibrido}
        \label{fig:vardim_tempo_H}
    \end{subfigure}
    \caption{Custo computacional em função da dimensão do problema sob o ponto de vista do tempo de processamento (a) SIMPLEX, (b) Pontos Interiores, (c) Híbrido.}
    \label{fig:vardim_tempo}
\end{figure}

\subsubsection{Número de iterações para a análise da variação do parâmetro valb para diferentes dimensões} 

\begin{figure}[H] 
    \centering
    \pgfplotstabletypeset
        [
            columns/valb/.style={column name=valb, fixed, precision=2},
            columns/2/.style={column name={n=2},fixed, precision=2},
            columns/4/.style={column name={n=4},fixed, precision=2},
            columns/6/.style={column name={n=6},fixed, precision=2},
            columns/10/.style={column name={n=10},fixed, precision=2},
            columns/14/.style={column name={n=14},fixed, precision=2},
            columns/18/.style={column name={n=18},fixed, precision=2},
            columns/20/.style={column name={n=20},fixed, precision=2},
            columns/22/.style={column name={n=22},fixed, precision=2},
            columns/24/.style={column name={n=24},fixed, precision=2},
        ]{datavalb/02_valb_iter_Simplex.txt}  
    \caption{Número de Iterações x Dimensão do problema para o algoritmo Simplex.}
\end{figure}

\begin{figure}[H] 
    \centering
    \pgfplotstabletypeset
        [
            columns/valb/.style={column name=valb, fixed, precision=2},
            columns/2/.style={column name={n=2},fixed, precision=2},
            columns/4/.style={column name={n=4},fixed, precision=2},
            columns/6/.style={column name={n=6},fixed, precision=2},
            columns/10/.style={column name={n=10},fixed, precision=2},
            columns/14/.style={column name={n=14},fixed, precision=2},
            columns/18/.style={column name={n=18},fixed, precision=2},
            columns/20/.style={column name={n=20},fixed, precision=2},
            columns/22/.style={column name={n=22},fixed, precision=2},
            columns/24/.style={column name={n=24},fixed, precision=2},
        ]{datavalb/02_valb_iter_PI.txt}  
    \caption{Número de Iterações x Dimensão do problema para o algoritmo de Pontos Interiores.}
\end{figure}

\begin{figure}[H] 
    \centering
    \pgfplotstabletypeset
        [
            columns/valb/.style={column name=valb, fixed, precision=2},
            columns/2/.style={column name={n=2},fixed, precision=2},
            columns/4/.style={column name={n=4},fixed, precision=2},
            columns/6/.style={column name={n=6},fixed, precision=2},
            columns/10/.style={column name={n=10},fixed, precision=2},
            columns/14/.style={column name={n=14},fixed, precision=2},
            columns/18/.style={column name={n=18},fixed, precision=2},
            columns/20/.style={column name={n=20},fixed, precision=2},
            columns/22/.style={column name={n=22},fixed, precision=2},
            columns/24/.style={column name={n=24},fixed, precision=2},
        ]{datavalb/02_valb_iter_Hibrido.txt}  
    \caption{Número de Iterações x Dimensão do problema para o algoritmo Híbrido.}
\end{figure}

\begin{figure}[H]
    \begin{subfigure}{0.45\textwidth}
        %\centering
        \def\filename{datavalb/02_valb_iter_Simplex.txt}  
        \begin{tikzpicture}[scale=0.85]
            \begin{axis}[
                xlabel={valb},
                ylabel={Iterações},
                title={valb x Iterações},
                ymode=log,
                %ymax=10^3,
                legend pos=outer north east,
                legend entries={{n=2},{n=4},{n=6},{n=10},{n=14},{n=18},{n=20},{n=22},{n=24}},
                ymajorgrids=true]
                \addplot table [x=valb,y=2]\filename ;
                \addplot table [x=valb,y=4]\filename ;
                \addplot table [x=valb,y=6]\filename ;
                \addplot table [x=valb,y=10]\filename ;
                \addplot table [x=valb,y=14]\filename ;
                \addplot table [x=valb,y=18]\filename ;
                \addplot table [x=valb,y=20]\filename ;
                \addplot table [x=valb,y=22]\filename ;
                \addplot table [x=valb,y=24]\filename ;
            \end{axis} 
        \end{tikzpicture}
        \caption{Simplex}
        \label{fig:vardim_tempo_S}
    \end{subfigure}
    \hfill
    \centering
    \begin{subfigure}{0.45\textwidth}
       % \centering
        \def\filename{datavalb/02_valb_iter_PI.txt}  
        \begin{tikzpicture}[scale=0.85] 
            \begin{axis}[
                xlabel={valb},
                ylabel={Iterações},
                title={valb x Iterações},
                ymode=log,
                ymin=10^0,
                ymax=10^2,
                legend pos=outer north east,
                legend entries={{n=2},{n=4},{n=6},{n=10},{n=14},{n=18},{n=20},{n=22},{n=24}},
                ymajorgrids=true]
                \addplot table [x=valb,y=2]\filename ;
                \addplot table [x=valb,y=4]\filename ;
                \addplot table [x=valb,y=6]\filename ;
                \addplot table [x=valb,y=10]\filename ;
                \addplot table [x=valb,y=14]\filename ;
                \addplot table [x=valb,y=18]\filename ;
                \addplot table [x=valb,y=20]\filename ;
                \addplot table [x=valb,y=22]\filename ;
                \addplot table [x=valb,y=24]\filename ;
            \end{axis} 
        \end{tikzpicture}
        \caption{PI}
        \label{fig:vardim_tempo_PI}
    \end{subfigure}
    \vfill
    \begin{subfigure}{0.45\textwidth}
        % \centering
        \def\filename{datavalb/02_valb_iter_Hibrido.txt}  
        \begin{tikzpicture}[scale=0.85] 
            \begin{axis}[
                xlabel={valb},
                ylabel={Iterações},
                title={valb x Iterações},
                ymode=log,
                ymin=10^0,
                ymax=10^2,
                legend pos=outer north east,
                legend entries={{n=2},{n=4},{n=6},{n=10},{n=14},{n=18},{n=20},{n=22},{n=24}},
                ymajorgrids=true]
                \addplot table [x=valb,y=2]\filename ;
                \addplot table [x=valb,y=4]\filename ;
                \addplot table [x=valb,y=6]\filename ;
                \addplot table [x=valb,y=10]\filename ;
                \addplot table [x=valb,y=14]\filename ;
                \addplot table [x=valb,y=18]\filename ;
                \addplot table [x=valb,y=20]\filename ;
                \addplot table [x=valb,y=22]\filename ;
                \addplot table [x=valb,y=24]\filename ;
            \end{axis} 
        \end{tikzpicture}
        \caption{Hibrido}
        \label{fig:vardim_tempo_H}
    \end{subfigure}
    \caption{Custo computacional em função da dimensão do problema sob o ponto de vista do tempo de processamento (a) SIMPLEX, (b) Pontos Interiores, (c) Híbrido.}
    \label{fig:vardim_tempo}
\end{figure}

\subsubsection{Erro percentual da função objetivo para a análise da variação do parâmetro valb para diferentes dimensões}

\begin{figure}[H] 
    \centering
    \pgfplotstabletypeset
        [
            columns/valb/.style={column name=valb, fixed, precision=2},
            columns/2/.style={column name={n=2},fixed, precision=2},
            columns/4/.style={column name={n=4},fixed, precision=2},
            columns/6/.style={column name={n=6},fixed, precision=2},
            columns/10/.style={column name={n=10},fixed, precision=2},
            columns/14/.style={column name={n=14},fixed, precision=2},
            columns/18/.style={column name={n=18},fixed, precision=2},
            columns/20/.style={column name={n=20},fixed, precision=2},
            columns/22/.style={column name={n=22},fixed, precision=2},
            columns/24/.style={column name={n=24},fixed, precision=2},
        ]{datavalb/03_valb_EFobj_Simplex.txt}  
    \caption{Erro percentual da função objetivo x Dimensão do problema para o algoritmo Simplex.}
\end{figure}

\begin{figure}[H] 
    \centering
    \pgfplotstabletypeset
        [
            columns/valb/.style={column name=valb, fixed, precision=2},
            columns/2/.style={column name={n=2},sci,sci zerofill,sci sep align,precision=2,sci superscript},
            columns/4/.style={column name={n=4},sci,sci zerofill,sci sep align,precision=2,sci superscript},
            columns/6/.style={column name={n=6},sci,sci zerofill,sci sep align,precision=2,sci superscript},
            columns/10/.style={column name={n=10},sci,sci zerofill,sci sep align,precision=2,sci superscript},
            columns/14/.style={column name={n=14},sci,sci zerofill,sci sep align,precision=2,sci superscript},
            columns/18/.style={column name={n=18},sci,sci zerofill,sci sep align,precision=2,sci superscript},
            columns/20/.style={column name={n=20},sci,sci zerofill,sci sep align,precision=2,sci superscript},
            columns/22/.style={column name={n=22},sci,sci zerofill,sci sep align,precision=2,sci superscript},
            columns/24/.style={column name={n=24},sci,sci zerofill,sci sep align,precision=2,sci superscript},
        ]{datavalb/03_valb_EFobj_PI.txt}  
    \caption{Erro percentual da função objetivo x Dimensão do problema para o algoritmo de Pontos Interiores.}
\end{figure}

\begin{figure}[H] 
    \centering
    \pgfplotstabletypeset
        [
            columns/valb/.style={column name=valb, fixed, precision=2},
            columns/2/.style={column name={n=2},fixed, precision=2},
            columns/4/.style={column name={n=4},fixed, precision=2},
            columns/6/.style={column name={n=6},fixed, precision=2},
            columns/10/.style={column name={n=10},fixed, precision=2},
            columns/14/.style={column name={n=14},fixed, precision=2},
            columns/18/.style={column name={n=18},fixed, precision=2},
            columns/20/.style={column name={n=20},fixed, precision=2},
            columns/22/.style={column name={n=22},fixed, precision=2},
            columns/24/.style={column name={n=24},fixed, precision=2},
        ]{datavalb/03_valb_EFobj_Hibrido.txt}  
    \caption{Erro percentual da função objetivo x Dimensão do problema para o algoritmo Híbrido.}
\end{figure}

\begin{figure}[H]
    \begin{subfigure}{0.45\textwidth}
        %\centering
        \def\filename{datavalb/03_valb_EFobj_Simplex.txt}  
        \begin{tikzpicture}[scale=0.85]
            \begin{axis}[
                xlabel={valb},
                ylabel={Erro [\%]},
                title={valb x Erro da função objetivo},
                legend pos=outer north east,
                legend entries={{n=2},{n=4},{n=6},{n=10},{n=14},{n=18},{n=20},{n=22},{n=24}},
                ymajorgrids=true]
                \addplot table [x=valb,y=2]\filename ;
                \addplot table [x=valb,y=4]\filename ;
                \addplot table [x=valb,y=6]\filename ;
                \addplot table [x=valb,y=10]\filename ;
                \addplot table [x=valb,y=14]\filename ;
                \addplot table [x=valb,y=18]\filename ;
                \addplot table [x=valb,y=20]\filename ;
                \addplot table [x=valb,y=22]\filename ;
                \addplot table [x=valb,y=24]\filename ;
            \end{axis} 
        \end{tikzpicture}
        \caption{Simplex}
        \label{fig:vardim_tempo_S}
    \end{subfigure}
    \hfill
    \centering
    \begin{subfigure}{0.45\textwidth}
       % \centering
        \def\filename{datavalb/03_valb_EFobj_PI.txt}  
        \begin{tikzpicture}[scale=0.85] 
            \begin{axis}[
                xlabel={valb},
                ylabel={Erro [\%]},
                title={valb x Erro da função objetivo},
                legend pos=outer north east,
                legend entries={{n=2},{n=4},{n=6},{n=10},{n=14},{n=18},{n=20},{n=22},{n=24}},
                ymajorgrids=true]
                \addplot table [x=valb,y=2]\filename ;
                \addplot table [x=valb,y=4]\filename ;
                \addplot table [x=valb,y=6]\filename ;
                \addplot table [x=valb,y=10]\filename ;
                \addplot table [x=valb,y=14]\filename ;
                \addplot table [x=valb,y=18]\filename ;
                \addplot table [x=valb,y=20]\filename ;
                \addplot table [x=valb,y=22]\filename ;
                \addplot table [x=valb,y=24]\filename ;
            \end{axis} 
        \end{tikzpicture}
        \caption{PI}
        \label{fig:vardim_tempo_PI}
    \end{subfigure}
    \vfill
    \begin{subfigure}{0.45\textwidth}
        % \centering
        \def\filename{datavalb/03_valb_EFobj_Hibrido.txt}  
        \begin{tikzpicture}[scale=0.85] 
            \begin{axis}[
                xlabel={valb},
                ylabel={Erro [\%]},
                title={valb x Erro da função objetivo},
                legend pos=outer north east,
                legend entries={{n=2},{n=4},{n=6},{n=10},{n=14},{n=18},{n=20},{n=22},{n=24}},
                ymajorgrids=true]
                \addplot table [x=valb,y=2]\filename ;
                \addplot table [x=valb,y=4]\filename ;
                \addplot table [x=valb,y=6]\filename ;
                \addplot table [x=valb,y=10]\filename ;
                \addplot table [x=valb,y=14]\filename ;
                \addplot table [x=valb,y=18]\filename ;
                \addplot table [x=valb,y=20]\filename ;
                \addplot table [x=valb,y=22]\filename ;
                \addplot table [x=valb,y=24]\filename ;
            \end{axis} 
        \end{tikzpicture}
        \caption{Hibrido}
        \label{fig:vardim_tempo_H}
    \end{subfigure}
    \caption{Custo computacional em função da dimensão do problema sob o ponto de vista do tempo de processamento (a) SIMPLEX, (b) Pontos Interiores, (c) Híbrido.}
    \label{fig:vardim_tempo}
\end{figure}

\subsubsection{Erro percentual da solução para a análise da variação do parâmetro valb para diferentes dimensões}

        \subsubsection{Conclusão}

        
    
        \newpage
        
    \subsection{Análise de sensibilidade quanto ao ponto inicial}
            
        \input{08_PontoInicialPar}
        
        \subsubsection{Conclusão}

        
    
        \newpage
\end{document}