%% Parte 1
\newcommand{\writetable}[2]
{ 
    \begin{figure}[!ht] 
    \centering
            \pgfplotstabletypeset
            [
                columns/freq/.style={column name=Dim, fixed, precision=1},
                columns/conc/.style={column name=Simplex,sci,sci zerofill,sci sep align,precision=1,sci superscript},
                columns/lino/.style={column name=PI,sci,sci zerofill,sci sep align,precision=1,sci superscript},
                columns/lino/.style={column name=Hibrido,sci,sci zerofill,sci sep align,precision=1,sci superscript},
            ]{#1}  
            \caption{#2 x Dimensão do problema.}
    \end{figure}
}

\newcommand{\plotline}[3]
{
    \begin{tikzpicture}[scale=0.6]
        \begin{axis}[
            xlabel={Dimensão},
            ylabel={Tempo de execução [S]},
            ymode=log,
            title={#3},
            legend pos=south east,
            legend entries={#2},ymajorgrids=true]
            \addplot table [x=Dim,y=#2]{#1};
        \end{axis}           
    \end{tikzpicture}
}

\section{Variação da Dimensão no espaço}
    \subsection{Tempo de execução X Dimensão do Problema}    
        \FloatBarrier
        \def \namefilepath {data/data01_tbl_dim_tempo.txt}
        \writetable{\namefilepath}{Tempo de execução}
        \begin{figure}[h!]
            \begin{subfigure}{0.3\textwidth}
                \centering
                \plotline{\namefilepath}{Simplex}{Tempo x Dimensão}
                \caption{Simplex}
                \label{fig:subim1}
            \end{subfigure}
            \hfill
            \begin{subfigure}{0.3\textwidth}
                \centering
                \plotline{\namefilepath}{PI}{Tempo x Dimensão}
                \caption{PI}
                \label{fig:subim2}
            \end{subfigure}
            \hfill
            \begin{subfigure}{0.3\textwidth}
                \centering
                \plotline{\namefilepath}{Hibrido}{Tempo x Dimensão}
                \caption{Hibrido}
                \label{fig:subim3}
            \end{subfigure}
            \caption{Custo computacional em função da dimensão do problema sob o ponto de vista do tempo de processamento (a) SIMPLEX, (b) Pontos Interiores, (c) Híbrido.}
            \label{fig:image2}
        \end{figure}
        \FloatBarrier
    \subsection{Número de iterações X Dimensão do Problema}    
        \FloatBarrier
        \def \namefilepath {data/data02_tbl_dim_N_Iter.txt}
        \writetable{\namefilepath }{Número de iterações}
        \begin{figure}[h!]
            \begin{subfigure}{0.3\textwidth}
                \centering
                \plotline{\namefilepath }{Simplex}{Iterações x Dimensão}
                \caption{Simplex}
                \label{fig:subim1}
            \end{subfigure}
            \hfill
            \begin{subfigure}{0.3\textwidth}
                \centering
                \plotline{\namefilepath }{PI}{Iterações x Dimensão}
                \caption{PI}
                \label{fig:subim2}
            \end{subfigure}
            \hfill
            \begin{subfigure}{0.3\textwidth}
                \centering
                \plotline{\namefilepath }{Hibrido}{Iterações x Dimensão}
                \caption{Hibrido}
                \label{fig:subim3}
            \end{subfigure}
            \caption{Custo computacional em função da dimensão do problema sob o ponto de vista do número de iterações (a) SIMPLEX, (b) Pontos Interiores, (c) Híbrido.}
            \label{fig:image2}
        \end{figure}        
        \FloatBarrier
    \subsection{Erro percentual da função objetivo X Dimensão do Problema}
        \writetable{data/data03_tbl_err_perc_fobj.txt}{Erro percentual da função objetivo}
    \subsection{Erro percentual da solução X Dimensão do Problema}
        \writetable{data/data04_tbl_err_perc_sol.txt}{Erro percentual da solução}
    \subsection{Conclusão}


    
    \newpage